\documentclass{assignment}

\usepackage{enumitem}
\usepackage{amssymb}
\usepackage{amsmath}
\usepackage{enumitem}
\usepackage[normalem]{ulem}

\title{CPSC 418 Assignment 3}
\coursetitle{Introduction to Cryptography}
\courselabel{CPSC 418}
\exercisesheet{Home Work \#3}{}
\student{Arnel Adviento - 10130641}
\semester{Fall 2016}
%\usepackage[pdftex]{graphicx}
%\usepackage{subfigure}
\usepackage{amsmath}

\begin{document}

\begin{center}
\renewcommand{\arraystretch}{2}
\begin{tabular}{|c|c|c|} \hline
Problem & Marks \\ \hline \hline
1 & \\ \hline
2 & \\ \hline
3 & \\ \hline
4 & \\ \hline
5 & \\ \hline
6 & \\ \hline
7 & \\ \hline \hline
Total & \\ \hline
\end{tabular}
\end{center}

\bigskip

\clearpage

\begin{problemlist}

\pbitem (Flawed hash function based MAC designs, 28 marks)
\begin{problem}
\begin{answer}

\begin{enumerate}[label=(\alph*)]
	\item \begin{enumerate}
    		\item We know
            	  \begin{align*}
    				M_1 \\
                    MAC_1 &= H(K||M_1') \\
                    M_2 &= M_1' || X 
  			 	  \end{align*}
                  We want to get
                  \begin{align*}
                  	MAC_2 &= H(K||M_2) \\
                    	  &= H(K||M_1'||X) 
                  \end{align*}
                  without knowing $K$
           		  \begin{align*}
                  	MAC_2 &= f( H(K||M_1')||X) \\
                    	  &= f(MAC_1 ||X)
                  \end{align*}
                  By doing this we showed we can compute the PHMAC of $M_2$ without know $K$ therefore defeating computation resistance. \\
            \item Since we know $H$ is not weakly collision resistance. There exists:
            	  \begin{align*}
            		M_1 &\neq M_2 \\
                    MAC_1 &= MAC_2
            	  \end{align*}
                  To generate a pair $(M_2, MAC_2)$ first we just pick an arbitrary message $M_2$. \\
                  We know that $MAC_2 = H(M_2'||K)$ even if we don't know $K$, since it is not weakly collision 				  resistant, we can find X which you append at the end of $M_2$ which will generate $MAC_2$ \\
		  \end{enumerate} 
  	\item \begin{enumerate}
    		\item we know
            	  \begin{align*}
                  	M_1 \\
            	  	MAC_1 &= E_K(M_1) \\
                    M_2 &= MAC_1 \\
                    MAC_2 &= E_K(M_2) \\ 
                    	  &= E_K(MAC_1) 
            	  \end{align*}
                  we want to know $MAC_3$ given a two-block message $M_3 = M_1||0^n$
                  \begin{align*}
                  	MAC_3 &= E_K(M_3) \\ 
                    	  %&= E_K(M_1||0^n) \\
                      C_1 &= E_K(M_1) \\
                      C_2 &= E_K(0^n \oplus E_K(M_1)) \\
                      	  &= E_K(E_K(M_1))\\
                          &= E_K(MAC_1) \\
                    MAC_3 &= MAC_2 \\ \\
					 M_3 &\neq M_2 \\ 
                     MAC_3 &= MAC_2 \\
                  \end{align*}
                  This violates computation resistance since it is not weakly collision resistance since there exists $MAC_3 = MAC_2$ but $M_3 \neq M_2$. \\
            \item we know
            	  \begin{align*}
            	  	M_1 \\
            	  	MAC_1 &= E_K(M_1) \\
                    M_2 \\
                    MAC_2 &= E_K(M_2) \\
                    M_3 &= M_1||X \\
                    MAC_3 &= E_K(M_3) \\
                    	  %&= E_K(M_1||X) \\
                      C_1 &= E_K(M_1) \\
                      C_2 &= E_K(X \oplus (E_K(M_1))) \\
                    MAC_3 &= E_K(X \oplus MAC_1) 
            	  \end{align*}
                  we want to know $MAC_4$ given a two-block message\\
                  $M_4 = M_2||(MAC_1 \oplus X \oplus MAC_2)$
                  \begin{align*}
                  	MAC_4 &= E_K(M_4)\\
                    	  &= E_K(M_2 ||(MAC_1 \oplus X \oplus MAC_2) \\
                      C_1 &= E_K(M_2) \\
                      C_2 &= E_K(MAC_1 \oplus X \oplus MAC_2 \oplus E_K(M_2) ) \\
                          &= E_K(MAC_1 \oplus X \oplus MAC_2 \oplus MAC_2 ) \\
                          &= E_K(MAC_1 \oplus X) \\ \\
                      M_4 &\neq M_3 \\
                      MAC_4 &= MAC_3 
                  \end{align*}
                  This violates computation resistance since it is not weakly collision resistance since there exists $MAC_4 = MAC_3$ but $M_4 \neq M_3$.
		  \end{enumerate}
\end{enumerate}

\end{answer}
\end{problem}

\clearpage
\pbitem (A modified man-in-the-middle attack on Die-Hellman, 12 marks)
\begin{problem}
\begin{answer}

\begin{enumerate}[label=(\alph*)]
	\item Bob receives $(g^a)^q$ from Mallory and computes $((g^a)^q)^b$ \\
    	  Alice receives  $(g^b)^q$ from Mallory and computes $((g^b)^q)^a$ \\ \\
          $((g^a)^q)^b \mod p = g^{aqb} \mod p = q^{bqa} \mod p = ((g^b)^q)^a \mod p$ \\ 
          based on exponent power rules
	\item We know that $g$ is a primitive root of $P$ and its smallest positive exponent is $k$ with $g^k \equiv 1$ (mod p) where $k = (p-1)$. We also know the equation $p = mq + 1$\\ \\
    So when Mallory raises $g$ to the power of $q$ this happens.
    	  \begin{align*}
          	g^q &= g^{(p-1) / m} (\bmod p)\\
            	&= g^{k/m} (\bmod p)
		  \end{align*}
    So we know that if we have $g$ has a key-space of $k$ and when we raise a power to $g$ for example $g^2$ we half the key-space, essentially getting $k/2$. So when we raise $g$ to the power of $q$ we get $g^q = g^{k/m}$, which mean we divide our key-space by $(k/m)$ which is $\frac{k}{(k/m)} = k \times \frac{m}{k} = m$ and we get a a new key-space of sized $m$.
	\item The man-in-the-middle attack discussed in class generates two different keys where Bob computes the key $(g^e)^b$ and Alice computes $(g^e)^a$. If Alice and Bob were to exchange the keys they computed they would notice that they would be different and will know that their keys have been tampered with. In this variation, we showed in part (a) that both Alice and Bob computed the same key and if they were to exchange those keys they would not notice the attack.   
\end{enumerate}

\end{answer}
\end{problem}


\pbitem (Binary exponentiation, 12 marks)
\begin{problem}
\begin{answer}

\begin{enumerate}[label=(\alph*)]
	\item Base Case (i = 0):
    	\begin{align*}
        	s_i &= \sum_{j=0}^i b_j 2^{i-j} \\
    		s_0 &= \sum_{j=0}^0 b_j 2^{0-j} \\
            	&= b_0 2^{0-0} \\
                &= b_0 \cdot 1 \\
            s_0 &= b_0
    	\end{align*}
        Inductive Hypothesis: \\
        Suppose $ s_i = \sum_{j=0}^i b_j 2^{i-j} $ we want to prove  $ s_{i+1} = 2s_i + b_{i+1} $
        \begin{align*}
        	s_{i+1} &= 2s_i + b_{i+1} \\
            		&= 2 (\sum_{j=0}^i b_j 2^{i-j}) + b_{i+1} \\
                    &= \sum_{j=0}^i b_j 2^{i-j+1} + b_{i+1} \\
                    &= (b_0 2^{i+1} + b_1 2^1 + b_2 2^{i-1} \ldots b_j 2^1) + b_{i+1} \\
                    &= (b_0 2^{i+1} + b_1 2^1 + b_2 2^{i-1} \ldots b_j 2^1) + b_{i+1} 2^0 \\
                    &= \sum_{j=0}^{(i+1)} b_j 2^{(i+1)-j}
        \end{align*}
	\item Base Case (i=0)
     	  \begin{align*}
     	  	r_i &\equiv a^{s_i} (\text{mod m}) \\
            r_0 &\equiv a^{s_0} (\text{mod m}) \\
            	&\equiv a^{b_0} (\text{mod m}) \\
                &\equiv a^1 (\text{mod m}) 
     	  \end{align*}
          Inductive Hypothesis: \\
          Suppose $ r_i \equiv a^{s_i} (\text{mod m})$ we want to prove  $ r_{i+1} = r_i^2 (\text{mod m})$ and $ r_{i+1} = r_i^2 a (\text{mod m})$ \newline \newline
          Case: $b_{i+1} = 0$
          \begin{align*}
          	r_{i+1} &\equiv r_i^2 (\text{mod m}) \\
            		&\equiv (a^{s_i})^2 (\text{mod m}) \\
                    &\equiv (a^{2s_i}) (\text{mod m})
          \end{align*}
          Case: $b_{i+1} = 1$
          \begin{align*}
          	r_{i+1} &\equiv r_i^2 a (\text{mod m}) \\
            		&\equiv (a^{s_i})^2 a (\text{mod m}) \\
                    &\equiv a^{(2s_i)} a (\text{mod m}) \\
                    &\equiv a^{(2s_i+1)} (\text{mod m})
          \end{align*}
	\item Since in part b) we proved that $r_i \equiv a^{s_i} (\bmod m)$ for $0 \leq i \leq k$ and in part a) we proved $s_i$ hold since we proved $s_{i+1}$ hold as well. Therefore $r_i \equiv a^{s_i} (\bmod m)$ does compute  $a^n (\bmod m)$ where the answer is $r_k (\bmod m)$.
\end{enumerate}

\end{answer}
\end{problem}

\clearpage
\pbitem (An RSA toy example for practicing binary exponentiation, 12 marks) \par Consider the RSA encryption scheme with public key (e, n) = (11, 77).
\begin{problem}
\begin{answer}


\begin{enumerate}[label=(\alph*)]
	\item \begin{align*}
			C &\equiv M^{e} \: (\text{mod n}) \\
            C &\equiv 17^{11} \: (\text{mod 77}) \\ \\
            17^{1} &= 17 \mod 77 \\
            17^{2} &= 58 \mod 77 \\
            17^{4} &= (17^2)^2 = 58^2 = 53 \mod 77 \\
            17^{8} &= (17^4)^2 = 53^2 = 37 \mod 77 \\ \\
            17^{11} &= 17^8 \cdot 17^2 \cdot 17^1 \\ 
            		&= 37 \cdot 58 \cdot 17 \\
                    &= 37 \cdot 58 \cdot 17 \\
                    &= 67 \cdot 17 \\
                    &= 61
		  \end{align*}
	\item \begin{align*}
			pq &= n \\
            pq &= 77 \\
            11 \times 7 &= 77 \\ \\
            de &\equiv 1 (\text{mod $\phi n$}) \\
            \phi n &= (p-1)(q-1) \\
            &= 10 \times 6 \\
            &= 60 \\ \\
            \text{Extended Euclidean} \\
            60 &= 5 \cdot 11 + 5 \\
            11 &= 2 \cdot 5 + 1 \\ \\
            1 &= 11 - 2 \cdot 5 \\
            &= 11 - 2 (60 \times (5 \cdot 11)) \\
            &= 11 - (2 \cdot 60) + (10 \cdot 11) \\
            &= 11(11) - (2)(60) \\
         	d &= 11 \\ \\
            de &\equiv 1 (\text{mod $\phi n$}) \\
            11 \cdot 11 &\equiv 1 (\text{mod 60}) 
		  \end{align*}
	\item \begin{align*}
			M &\equiv C^d \: (\text{mod n}) \\
            &\equiv 32^{11}  (\text{mod 77}) \\
            \text{binary of 11} &= 1011 \\
            %&\equiv 65
            r_0 &= 32 \: \text{mod 77} \\ 
            r_1 &= 32^2 \: \text{mod 77} = 23 \: \text{mod 77} \\
            r_2 &= 23^2 \cdot 32  \: \text{mod 77} = 65 \: \text{mod 77} \\
            r_3 &= 65^2 \cdot 32  \: \text{mod 77} = 65 \: \text{mod 77} \\
            M &= 65
		  \end{align*}
\end{enumerate}

\end{answer}
\end{problem}

\end{problemlist}
\end{document}